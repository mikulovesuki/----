\documentclass{article}
\usepackage{ctex}
\usepackage{mathtools}
\usepackage{amsthm}
\usepackage{amsfonts}
\usepackage{float}
\usepackage{tikz}
\usepackage{graphicx}
\usepackage{indentfirst} % 使首段也缩进
\usepackage{forest}
\setlength{\parindent}{2em} % 设置首行缩进为两个汉字宽度
\everymath{\displaystyle}
\title{}
\author{}
\date{}
\begin{document}
\maketitle
\begin{center}
\section*{摘要}
\end{center}


关键词:\textit{关键词1},\textit{关键词2},\textit{关键词3}

\section{问题复述}

\subsection{问题背景}

生鲜商超中的蔬菜类商品具有显著的易腐性特征,其保鲜期通常极短,且品相会随销售时间的增加而持续变差。对于大多数蔬菜品种而言,如果当日未能售出,次日便无法继续销售,这直接导致了高昂的损耗风险和对每日精准补货的迫切需求 。为了应对这一挑战,商超通常会根据各类商品的历史销售和需求情况进行每日补货。   

然而,补货决策的制定面临多重复杂性。蔬菜的进货交易时间通常在凌晨3:00至4:00之间,这意味着商家必须在不确切了解具体单品和其进货价格的情况下,做出当日各蔬菜品类的补货决策 。这种固有的不确定性构成了决策过程中的核心挑战,我们需要结合附件中各品种的各方面数据,建立一个有效的模型来指导商超的补货决策。



\subsection{问题一}

蔬菜类商品的不同种类间可能存在一种内在联系,例如某些蔬菜可能在销售上存在着竞争、替补或相互依存的关系。我们需要对各蔬菜品种的销售分布进行分析,得出各蔬菜品种之间的规律及其相互关系。

\subsection{问题二}

蔬菜类商品通常以品类为单位进行补货决策。为了得到最大收益,实现最优补货决策,需要我们对蔬菜的销售情况与成本加成定价的关系进行平衡。以过往数据为基础,为未来一周的蔬菜品类的日进货总量和定价策略给出最优方案。
\subsection{问题三}

根据过去一周的销售品类,在销售单品数在27到33,且各订单订购量大于等于2.5千克的条件下,给出7月1日各蔬菜单品补货量及其定价策略,以实现最大化的收益。
\subsection{问题四}

分析除批发价格、损耗率、品类竞争关系之外影响蔬菜商品补货和定价决策的其他因素。通过对这些因素的深入研究,提出更全面的补货和定价策略。
\section{问题分析}

\subsection{问题一分析}

\subsection{问题二分析}

\subsection{问题三分析}

\subsection{问题四分析}

\section{符号说明}

\section{数据预处理}

\section{模型假设}

\section{模型求解}


\subsection{问题一模型求解}

\subsection{问题二模型求解}

\subsection{问题三模型求解}

\subsection{问题四模型求解}


\section{模型检验}

\section{模型优点和展望}



\end{document}