\documentclass{article}
\usepackage{ctex}
\usepackage{mathtools}
\usepackage{amsthm}
\usepackage{amsfonts}
\usepackage{float}
\usepackage{tikz}
\usepackage{graphicx}
\usepackage{indentfirst} % 使首段也缩进
\usepackage{forest}
\setlength{\parindent}{2em} % 设置首行缩进为两个汉字宽度
\everymath{\displaystyle}
\title{基于浮动价格下的蔬菜商品补货与自动定价决策模型}
\author{}
\date{}
\begin{document}
\maketitle
\begin{center}
\section*{摘要}
\end{center}


关键词:\textit{关键词1},\textit{关键词2},\textit{关键词3}

\section{问题复述}

\subsection{问题背景}

生鲜商超中的蔬菜类商品具有显著的易腐性特征,其保鲜期通常极短,且品相会随销售时间的增加而持续变差。对于大多数蔬菜品种而言,如果当日未能售出,次日便无法继续销售,这直接导致了高昂的损耗风险和对每日精准补货的迫切需求 。为了应对这一挑战,商超通常会根据各类商品的历史销售和需求情况进行每日补货。   

然而,补货决策的制定面临多重复杂性。蔬菜的进货交易时间通常在凌晨3:00至4:00之间,这意味着商家必须在不确切了解具体单品和其进货价格的情况下,做出当日各蔬菜品类的补货决策 。这种固有的不确定性构成了决策过程中的核心挑战,我们需要结合附件中各品种的各方面数据,建立一个有效的模型来指导商超的补货决策。



\subsection{问题一}

蔬菜类商品的不同种类间可能存在一种内在联系,例如某些蔬菜可能在销售上存在着竞争、替补或相互依存的关系。我们需要对各蔬菜品种的销售分布进行分析,得出各蔬菜品种之间的规律及其相互关系。

\subsection{问题二}

蔬菜类商品通常以品类为单位进行补货决策。为了得到最大收益,实现最优补货决策,需要我们对蔬菜的销售情况与成本加成定价的关系进行平衡。以过往数据为基础,为未来一周的蔬菜品类的日进货总量和定价策略给出最优方案。
\subsection{问题三}

根据过去一周的销售品类,在销售单品数在27到33,且各订单订购量大于等于2.5千克的条件下,给出7月1日各蔬菜单品补货量及其定价策略,以实现最大化的收益。
\subsection{问题四}

分析除批发价格、损耗率、品类竞争关系之外影响蔬菜商品补货和定价决策的其他因素。通过对这些因素的深入研究,提出更全面的补货和定价策略。
\section{问题分析}

\subsection{问题一分析}
问题一要求我们对各个品类的单品销售量之间的潜在相互关系。本研究对各蔬菜商品品类的时间、销售分布的方向,探究各品类销量随时间的分布规律(应季蔬菜),随其他品类蔬菜销售量的分布规律(存在竞争或依存关系的蔬菜)和随商品定价的分布规律三个角度进行分析。
其中相互关系可以分为品类间的相互关系和单品间的相互关系。从这几个角度出发,我们可以较为清楚直观地得到销售量的分布规律和相互关系。除此之外,通过对销售量分布规律和相互关系的研究也可以辅助我们得到到销售量的影响因素。结合生活中的常识以及经济学理论,可以提高对影响因素分析的准确性。

\subsection{问题二分析}


\subsection{问题三分析}

\subsection{问题四分析}

\section{符号说明}

$ Q $日销售量
\section{数据预处理}

\subsection{数据的整合与排查}
为确保后续模型建立和计算的准确性和可靠性,对原始数据进行彻底的排查和整合十分重要。本研究将利用附件1(商品信息)、附件2(销售流水明细)、附件3(蔬菜批发价格)和附件4(商品近期损耗率)中提供的全部数据
对其中的缺失值、异常值进行处理和排除,以便后续分析。
\begin{itemize}
    \item 异常值的处理:从附件中提取各个关键的数据,将数据按“销售日期”、“品类”和“单品名称”进行聚合,计算每日的销售总量和总金额,并进一步计算每日的平均销售单价。采用Z-Score方法检测异常值,设定阈值为3,剔除超过该阈值的异常数据。
    \[
\underset{\text{Z-Score方法公式}}{Z = \frac{x - \mu}{\sigma}}
\]

    \item 无关值的处理:观察蔬菜商品销售数据,注意到部分数据难以找到内在联系,无法观察出其周期性规律,并且数据样本过少或过于离散,没有呈现出一定的客观规律。本研究认为此类蔬菜商品属于特殊蔬菜品种,不属于主流蔬菜品种,供应量极少或不属于本地品种,在本地市场需求少,没有稳定的供需关系和销售目标,\textit{此类蔬菜商品对蔬菜销量的预测和商品定价的决策没有有利贡献,因此本研究认为此类数据是无关数据,做剔除处理。}
    
    \item 不同参量的分类:按时间、价格为参量,对数据进行整合和分类,
\end{itemize}

\subsection{数据处理结果}


\section{模型假设}

\subsection{问题一的模型假设}

\subsubsection{确立对象化模型}
对问题一的分析,我们确立了三个不同角度来对各蔬菜品类商品的分布规律和相互联系进行研究,这样能综合各方面影响因素,结合最多的数据得到较为准确的结论,分别是

\begin{itemize}
    \item \textit{时间分布规律}:附件中给出的各蔬菜商品销售数据
    
    \item \textit{商品定价分布规律}:
    
    \item \textit{各蔬菜品种间销售量分布规律}:
\end{itemize}

\subsubsection{随时间分布规律的研究}

\subsubsection{随商品定价分布规律的研究}

\subsubsection{随其他品类蔬菜销售量的分布规律的研究}
\section{模型求解}


\subsection{问题一模型求解}

\subsection{问题二模型求解}

\subsection{问题三模型求解}

\subsection{问题四模型求解}


\section{模型检验}

\section{模型优点和展望}



\end{document}